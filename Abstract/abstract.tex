% ************************** Thesis Abstract *****************************
% Use `abstract' as an option in the document class to print only the titlepage and the abstract.
\begin{abstract}
Dimensionality Reduction (DR) algorithms such as t-SNE intend to produce 2D or 3D visual embeddings from high-dimensional data. However, despite embeddings are simple and beautiful data representations, it has been shown that non-expert Machine Learning (ML) users can derive wrong conclusions if they don't select the most appropriate hyper-parameters for fitting the algorithm to that specific dataset. In a similar way, groups of attributes and instances the could  represent, for instance, high-levels of noise in the data can significantly affect the resulting embedding. To address this, we present a web-based tool designed for domain-expert users, which don't need to be experts in ML and even have programming skills to use. In the same way than other DR-based Visual Analytics tools, our tool allows users to explore the hyper-parameter space while interactively seeing changes in the embedding. In addition, we provide users with the ability to understand what changes in the original high-dimensional space, in terms of attributes and instances, drive the fitting of the embedding space. Going further, this tool enable domain-expert users to perform cluster-oriented DR task sequences for their own labeled or unlabeled data, providing a implementation of the K-Means clustering algorithm, without having to write a line of code. This paper presents the design process of the tool, which takes some recommendations from the state of the art for building interactive ML interfaces, and presents two case studies that demonstrate its functionalities. We also contribute the open-source tool that is accessible for domain-experts to derive insights from their data.
\end{abstract}

% Contributions: 
% 1. Summarize state of the art for Interactive
% 2. Summarize state of the art for Interpretable
% 3. Contrast those sub-fields with the classic ML
% 4. Describe guidelines for designing Interpretable and Interactive ML systems