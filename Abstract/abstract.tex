% ************************** Thesis Abstract *****************************
% Use `abstract' as an option in the document class to print only the titlepage and the abstract.
\begin{abstract}
In Exploratory Data Analysis (EDA), Machine Learning (ML) is an alternative for understanding larger and high-dimensional data. Modern Dimensionality Reduction (DR) algorithms such as t-SNE and UMAP produce two or three dimensional embeddings (scatter plots) looking to preserve local and global structure of the data. By the other hand, Clustering algorithms such as K-Means and DBSCAN seek to achieve a similar goal by producing a cluster for each data instance. In general terms, when using these kind of algorithms, non-expert ML users can derive wrong conclusions if an appropriate set of hyper-parameters for fitting the algorithm to that specific dataset is not selected. Similarly, groups of attributes (or dimensions in the original space) and  data instances could represent, for instance, high-levels of noise in the data significantly affecting the embedding and clustering formations. To address this, a web-based tool for exploring high-dimensional tabular data that implements the t-SNE and K-Means algorithms running in the browser is presented. Because this tool is targeted to domain-expert users, which do not need to be experts in ML and even have programming skills, some concepts and recommendations for designing user-centric ML models and systems are derived from Interactive ML and Interpretable ML sub-fields. Like other ML-based Visual Analytics (VA) tools, this allows users to explore the hyper-parameter space while interactively seeing how these changes affect the model results. In addition, the ability to evidence model changes when user perform feature selection and instance filtering is also included. Going further, this tool enable domain-expert users to perform cluster-oriented DR task sequences such as cluster verification, cluster naming and cluster and classes matching. To evaluate the tool, two case studies demonstrating its usefulness for exploring real-world datasets are also presented. Finally, the tool is published under MIT license for open usage by academical and practitioner users taking advantage of the fact that backend is not required for model training.

%Preserve the global structure, in the context of these algorithms, is established as the ability for making clusters visually identifiable by the user.

%Dimensionality Reduction (DR) algorithms such as t-SNE intend to produce 2D or 3D visual embeddings from high-dimensional data. However, despite embeddings are simple and beautiful data representations, it has been shown that non-expert Machine Learning (ML) users can derive wrong conclusions if they don't select the most appropriate hyper-parameters for fitting the algorithm to that specific dataset. In a similar way, groups of attributes and instances the could  represent, for instance, high-levels of noise in the data can significantly affect the resulting embedding. To address this, we present a web-based tool designed for domain-expert users, which don't need to be experts in ML and even have programming skills to use. In the same way than other DR-based Visual Analytics tools, our tool allows users to explore the hyper-parameter space while interactively seeing changes in the embedding. In addition, we provide users with the ability to understand what changes in the original high-dimensional space, in terms of attributes and instances, drive the fitting of the embedding space. Going further, this tool enable domain-expert users to perform cluster-oriented DR task sequences for their own labeled or unlabeled data, providing a implementation of the K-Means clustering algorithm, without having to write a line of code. This paper presents the design process of the tool, which takes some recommendations from the state of the art for building interactive ML interfaces, and presents two case studies that demonstrate its functionalities. We also contribute the open-source tool that is accessible for domain-experts to derive insights from their data.
\end{abstract}

% Contributions: 
% 1. Summarize state of the art for Interactive
% 2. Summarize state of the art for Interpretable
% 3. Contrast those sub-fields with the classic ML
% 4. Describe guidelines for designing Interpretable and Interactive ML systems